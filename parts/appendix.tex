\newpage

\appendix

\section*{Appendix: \quad Metazoan segmentation example}

As a biologically inspired example, we propose here the model of \pref{fig:metazoan-nocoop} given in~\cite{PMR10-TCSB} of the metazoan segmentation network.
This PH models three components containing two processes each.
A controller gene $f$ activates the two others;
the products of $a$ are responsible for a new pattern, while gene $c$ tends to inhibit $a$, thus removing the pattern.
This negative feedback therefore leads to a sequence of activations and inhibitions of $a$, creating stripes.

When applying the IG inference described in \pref{sec:infer-IG}, we obtain the IG given in \pref{fig:metazoan-BRN}(left).
All the edges of this inferred IG exist in the original IG of~\cite{PMR10-TCSB} which was used to produce the PH.
However, an edge $a \xrightarrow{+1} a$ is present in the original IG but is not found by our method.
This can be explained by the absence of actions in the PH to model this self-influence.
Furthermore, some self-influences model phenomena that impact the inferred parametrization rather than the edges of the IG
%are rather inferred as parameters by our method than as edges in the IG
(such as the action $\PHfrappe{f_1}{f_1}{f_0}$ which does not results in a $f \xrightarrow{-1} f$ edge).

Then, the parameters inference presented in \pref{sec:infer-K} allows to infer, from the input PH of \pref{fig:metazoan-nocoop} and inferred IG of \pref{fig:metazoan-BRN}(left), the parameters given in \pref{fig:metazoan-BRN}(right).
Some of the parameters, which are signaled with an interrogation mark, could not be inferred due to contradicting influences from the regulators.
For instance, in context $\{c_0;f_1\}$, both actions $\PHfrappe{c_0}{a_0}{a_1}$ and $\PHfrappe{f_1}{a_1}{a_0}$ apply and result in a cycle of bounces between $a_0$ and $a_1$.
Thus, no focal process is found in sort $a$, which explains why the parameter $K_{a,\{c\},\{f\}}$ is unknown.
It is interesting to notice here that the action $\PHfrappe{f_1}{f_1}{f_0}$ previously mentioned simply results in the parameter value $K_{f,\emptyset,\emptyset} = [0;0]$.

Because two parameters are impossible to infer, the original PH of \pref{fig:metazoan-nocoop} does not correspond to one unique BRN, but to a family of BRNs sharing the same IG but different parametrizations.
This result and the method to find all admissible parametrizations of this family are given by \pref{ssec:admissible-K}.
As both unknown parameters can take one value amongst $[0;0]$, $[1;1]$ and $[0;1]$, the family contains 9 different BRNs.

By creating a cooperative sort involving $f$ and $c$, it is possible to refine the dynamics and avoid the concurrent actions that prevent all parameter inferences.
Such a cooperative sort is described in the reference paper, and allows to infer a complete parametrization, thus matching the original PH to a unique BRN.

\begin{figure}[p]
\centering
\scalebox{1.1}{
\begin{tikzpicture}
\path[use as bounding box] (-1,-1.5) rectangle (4,4);

\TSort{(1,3)}{f}{2}{t}
\TSort{(0,0)}{c}{2}{l}
\TSort{(3,0)}{a}{2}{r}

\THit{c_0}{selfhit, out=250, in=210, distance=30}{c_0}{.west}{c_1}
\THit{c_1}{selfhit, out=120, in=150, distance=30}{c_1}{.west}{c_0}
\THit{c_0}{}{a_0}{.west}{a_1}
\THit{c_1}{}{a_1}{.west}{a_0}
\path[bounce, bend left=60]
\TBounce{c_0}{}{c_1}{.south west}
\TBounce{c_1}{bend right=40}{c_0}{.north west}
\TBounce{a_0}{bend left=60}{a_1}{.south west}
\TBounce{a_1}{bend right=40}{a_0}{.north west}
;

\THit{f_1}{selfhit, out=50, in=120, distance=40}{f_1}{.north west}{f_0}
\THit{f_1}{bend right, out=120, in=80, distance=150}{c_0}{.south east}{c_1}
\THit{f_0}{}{c_1}{.north east}{c_0}
\THit{f_1}{bend right, out=50, in=110, distance=50}{a_0}{.east}{a_1}
\THit{f_0}{}{a_1}{.north}{a_0}
\path[bounce, bend left=25]
\TBounce{f_1}{bend right=50}{f_0}{.north}
\TBounce{c_1}{bend left=50}{c_0}{.north east}
\TBounce{c_0}{bend right=50}{c_1}{.south east}
\TBounce{a_1}{bend left=50, out=100, in=120, distance=14}{a_0}{.north east}
\TBounce{a_0}{bend right=40}{a_1}{.south east}
;
\end{tikzpicture}
}

\caption{\label{fig:metazoan-nocoop}
The PH model of metazoan segmentation process. This model contains three components ($a$, $c$ and $f$) but no cooperative sort,
leading to concurrent actions on $a$, such as: $\PHfrappe{f_1}{a_0}{a_1}$ and $\PHfrappe{c_1}{a_1}{a_0}$.
}
\end{figure}



\begin{figure}[p]
\begin{minipage}{0.4\linewidth}
\centering
\scalebox{1.2}{
\begin{tikzpicture}[grn]
\path[use as bounding box] (-1,-0.2) rectangle (2.5,1.5);
\node[inner sep=0] (c) at (0,0) {c};
\node[inner sep=0] (a) at (2,0) {a};
\node[inner sep=0] (f) at (1,1.2) {f};
%\path
%  node[elabel, below=-1em of a] {$0..2$}
%  node[elabel, below=-1em of b] {$0..1$}
%  node[elabel, below=-1em of c] {$0..1$};
\path[->]
  (f) edge node[elabel, above right=-2pt] {$+1$} (a)
  (f) edge node[elabel, above left=-2pt] {$+1$} (c)
  (c) edge node[elabel, above=-4pt] {$-1$} (a)
  (c) edge[loop left] node[elabel, left=-2pt] {$-1$} (c);
%\path[->]
%  (b) edge[bend right] node[elabel, below=-2pt] {$+1$} (a)
%  (c) edge node[elabel, right=-2pt] {$+1$} (a)
%  (a) edge[bend right] node[elabel, above=-5pt] {$-2$} (b);
\end{tikzpicture}
}
\end{minipage}
\begin{minipage}{0.6\linewidth}
\centering
\begin{align*}
K_{f, \emptyset, \emptyset} & =  [0;0] \\
K_{a, \emptyset, \{c;f\}} & =  [0;0] & K_{c, \emptyset, \{c;f\}} & =  [0;0] \\
K_{a, \{f\}, \{c\}} & = \ \ \ ? & K_{c, \{c\}, \{f\}} & =  [1;1] \\
K_{a, \{c\}, \{f\}} & = \ \ \ ? & K_{c, \{f\}, \{c\}} & =  [0;0] \\
K_{a, \{c;f\}, \emptyset} & =  [1;1] & K_{c, \{c;f\}, \emptyset} & =  [1;1] \\
\end{align*}
\end{minipage}
\caption{\label{fig:metazoan-BRN}
(left)
IG inferred from the PH model of metazoan segmentation given in \pref{fig:metazoan-nocoop}.
(right)
Parameters inferred from the PH model of \pref{fig:metazoan-nocoop} together with the (left) IG.
The interrogation marks indicate parameters that could not be inferred due to the expression of opposite influences.
}
\end{figure}
