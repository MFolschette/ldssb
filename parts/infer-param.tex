\section{Parametrization inference}\label{sec:infer-K}

Given the IG inferred from a PH as presented in the previous section, one can find the discrete parameters that model the behavior of the studied PH using the method presented in the following.
As some parameters may remain undetermined, another step allows to enumerate all parametrizations compatible with the inferred parameters.

%It relies on an exhaustive enumeration of all predecessors of each component in order to find attractor processes, and returns a possibly incomplete parametrization given the exhaustiveness of the cooperations.
%The last step consists of the enumeration of all compatible complete parametrizations given this set of inferred parameters, the PH dynamics and some biological constraints on parameters.

\subsection{Independent parameters inference}

This subsection presents some results related to the inference of independent discrete parameters from a given PH,
equivalent to those presented in \cite{PMR10-TCSB}.
We suppose in the following that the considered PH is well-formed for parameters inference: its inferred IG does not contain any unsigned edge,
and in each sort, all processes activating (resp inhibiting) another component share the same behavior.
%In addition, we introduce the well-formed PH for parameter inference property (\pref{pro:wf-ph-K}),
%which implies that the inferred IG does not contain any unsigned interactions, and thus can be seen as a regular IG $(\Gamma, E_+, E_-)$,
%and that in each sort, all processes activating (resp inhibiting) another component share the same behavior.
%and that any processes in $\levelsA{b}{a}$ (resp. $\levelsI{b}{a}$) share the same behavior regarding $a$.
Let $K_{a,A,B}$ be the parameter we want to infer for a given component $a \in \Gamma$
%and $A,B \in \GRNallres{a}$ a configuration of resources of $a$ (activators and inhibitors).
and $A \subset \GRNreg{a}$ (resp. $B \subset \GRNreg{a}$) a set of its activators (resp. inhibitors).
This inference, as for the IG inference, relies on the search of focal processes of the component for the given configuration of its regulators.
%However, the considered context is not strict.

For each sort $b \in \GRNreg{a}$, we define a context %$C^b_{a,A,B}$ in \pref{eq:param_context}
that represents the influence of the regulators in the configuration $A,B$ (including the cooperative sorts involved).
%The context of a cooperative sort $\upsilon$ that regulates $a$ is given in
%\pref{eq:param_context_coop} as the set of focal processes matching the current configuration.
%$C_{a,A,B}$ refers to the union of all these contexts (\pref{eq:K-ctx}).
The parameter $K_{a,A,B}$ specifies to which values $a$ eventually evolves as long as the context
holds, which is precisely given by the set of focal processes. %$\focals$ function (\pref{def:focals}),
%where the focals reachability property can be derived from \pref{pro:wf-ph-K} and
%\pref{eq:param_context_coop}.
%Hence $K_{a,A,B} = \focals(a,C^a_{a,A,B},C_{a,A,B})$ if this latter is a non-empty interval
%(\pref{pps:param_K}).

\begin{example*}
Applied to the PH in \pref{fig:runningPH}, we obtain, in particular,
$K_{b,\{a\},\emptyset} = [0 ; 1]$,
$K_{a,\{b,c\},\emptyset} = [2 ; 2]$ and
$K_{a,\{b\},\{c\}} = [1;1]$.
\end{example*}

This method sometimes faces cases with opposite effects on a component, leading to either an indeterministic evolution or to oscillations.
Such an indeterminism is not possible in a BRN, and the inference of the targeted parameter is impossible.
%as in a given configuration of regulators, a component can only have an interval attractor, and eventually reaches a steady-state.
In order to avoid such inconclusive cases, one has to ensure that no such behavior is allowed
by either removing undesired actions or using cooperative sorts to prevent opposite influences between regulators.

\subsection{Admissible parametrizations}\label{ssec:admissible-K}

In the following, we try to constrain all parameters that are left undetermined with the method presented in the previous subsection.
We consider that a parameter is valid if any transition it involves in the resulting BRN is allowed by the studied PH by actions that represent this behavior.
%We can delimit the validity of a parameter by ensuring that any transition it involves in the resulting BRN is allowed by the studied PH, \ie there exists a bounce in the right direction.
We also add some biological constraints on the whole parametrizations, given in \cite{BernotSemBRN}.
These constraints lead to a family of admissible parametrizations which we can enumerate and are ensured to observe a coherent behavior that is included in the original PH.


\begin{comment}
When building a BRN, one has to find the parametrization that best describes the desired behavior of the studied system.
Complexity is inherent to this process as the number of possible parametrizations for a given IG is exponential w.r.t. the number of components.
However, the method of parameters inference presented in the previous subsection gives some information about necessary parameters given a certain dynamics described by a PH.
This information thus drops the number of possible parametrizations, allowing to find the desired behavior more easily.

We first delimit the validity of a parameter (\pref{pro:K-valid}) in order to ensure that any
transition in the resulting BRN is allowed by the studied PH.
This is verified by the existence of a hit making the concerned component bounce into the direction
of the value of the parameter in the matching context.
Thus, assuming \pref{pro:wf-ph-K} holds, any transition in the inferred BRN corresponds to at least
one transition in the PH, proving the correctness of our inference.
We remark that any parameter inferred by \pref{pps:param_K} satisfies this property.
Then, we use some additional biological constraints on Thomas' parameters given in
\cite{BernotSemBRN}, that we sum up in Properties \pref{prop:param_enum_extreme},
\pref{prop:param_enum_activity} and \pref{prop:param_enum_monotonicity}.
\end{comment}

This approach can be considered as abductive reasoning as some information is added by the enumeration.
If we denote:
\begin{itemize}
  \item $M$ the resulting BRN that observes a behavior included into the one of the original PH,
  \item $B$ the IG and the series of necessary parameters inferred from the original PH,
  \item $H_K$ the hypothesis that we select the complete parametrization $K$
\end{itemize}
The set of parametrizations $K$ that answers our expectations are the ones so that:
\begin{itemize}
  \item $H_K$ is compatible with $B$, that is, all parameters of $K$ are compatible with the inferred parameters in $B$,
  \item $B \wedge H_K \models M$, that is, the parametrization $K$ found from enumeration and the inferred IG represent a BRN observing the behavior included into the behavior of the original PH.
\end{itemize}

Answer Set Programming (ASP) \cite{Baral03} turns out to be effective for the enumerative searches developed in this paper,
as it efficiently tackles with the inherent complexity of the models we use, thus allowing an efficient execution of the formal tools developed. %, and is convenient to enumerate large sets of possible answers.
Furthermore, ASP finds a particularly interesting application in the research of admissible parametrizations regarding the properties presented above, as this enumeration can be naturally formulated with the use of aggregates, and constraints allow to remove all non-admissible models.



\begin{comment}
\subsection{Answer Set Programming implementation concepts}

\newcommand{\ti}[1]{\texttt{\textit{#1}}}
\newcommand{\aspil}[1]{\texttt{#1}}
\newcommand{\asp}[1]{\begin{itemize} \item[] \aspil{#1} \end{itemize}}


All information describing the studied model (PH and inferred IG \& parameters) can be expressed in ASP using facts.
For functional purposes, we assign a unique label to each couple $A,B$ of activators and inhibitors of a given component, which allows to refer to the related parameter (in the following, we note $K^p_{a,A,B}$ the parameter of component $a$ whose regulators $A,B$ are assigned to the label $p$).
Consequently, to express that it exists a parameter of component \ti{a} with the label \ti{p}, we use an atom named \aspil{param\_label} in the following fact:
\asp{param\_label(\ti{a}, \ti{p}).}

Defining a set in ASP is equivalent to defining the rule for belonging to this set. For example, we can define an atom \aspil{param\_act} that describes the set of all active regulators for a parameter of component \ti{a} and label \ti{p} (\ie the set $A$ of a parameter $K^\ti{p}_{\ti{a},A,B}$). For example, describing the activators of $K^\ti{p}_{\ti{a},\{\ti{b},\ti{c}\},\{\ti{d}\}}$ gives:
\asp{param\_act(\ti{a}, \ti{p}, \ti{b}).
\item[] param\_act(\ti{a}, \ti{p}, \ti{c}).}
The absence of such a fact involving \ti{d} with label \ti{p} indicates that \ti{d} is an inhibitor in the configuration of regulators related to this parameter.

Rules allow more detailed declarations than facts as they have a body (right-hand part below) containing constraints and allowing to use variables, while facts only have a head (left-hand part).
For instance, in order to define the set of expression levels of a component, we can declare:
\asp{component\_levels(X, 0..M) :- component(X, M).}
where the \aspil{component(X, M)} atom stands for the existence of a component \aspil{X} with a maximum level \aspil{M}.
Considering this declaration, any possible answer for the atom \aspil{component\_levels} will be found by binding all possible values of its terms with all existing \aspil{component} facts: an answer \aspil{component\_levels(\ti{a}, \ti{k})} will depend on the existence of a term \ti{a}, which is bound with \aspil{X}, and an integer~\ti{k}, constrained by: $0 \leq \ti{k} \leq \aspil{M}$.

Cardinalities are convenient to enumerate all possible parametrizations by creating multiple answer sets.
A cardinality (denoted hereafter with curly brackets) gives any number of possible answers for some atoms between a lower and upper bounds.
For example,
\asp{1 \{ param(X, P, I) : component\_levels(X, I) \} :-
\item[] ~~~~~~param\_label(X, P), not infered\_param(X, P).}
where \aspil{param(X, P, I)} stands for: $\aspil{I} \in K^\aspil{P}_{\aspil{X},A,B}$,
means that any parameter of component \aspil{X} and label \aspil{P} must contain at least one level value (\aspil{I}) in the possible expression levels of \aspil{X}.
Indeed, the lower bound is 1, forcing at least one element in the parameter, but no upper bound is specified, allowing up to any number of answers.
The body (right-hand side) of the rule also checks for the existence of a parameter of \aspil{X} with label \aspil{P}, and constrains that the parametrization inference was not conclusive for the considered parameter (\aspil{not} stands for negation by failure: \aspil{not L} becomes true if \aspil{L} is not true).
Such a constraint gives multiple results as any set of \aspil{param} atoms satisfying the cardinality will lead to a new global set of answers.
In this way, we can enumerate all possible parametrizations which respects the results of parameters
inference, but completely disregarding the notion of admissible parametrizations given in
\pref{ssec:admissible-K}.

We rely on integrity constraints to filter only admissible parametrizations.
An integrity constraint is a rule with no head, that makes an answer set unsatisfiable if its body turns out to be true.
Hence, supposing that:
\begin{itemize}
  \item the \aspil{less\_active(\ti{a}, \ti{p}, \ti{q})} atom means that $K^\ti{p}_{\ti{a},A,B}$ stands for a configuration with less activating regulators than $K^\ti{q}_{\ti{a},A',B'}$ (\ie $A \subset A'$),
  \item the \aspil{param\_inf(\ti{a}, \ti{p}, \ti{q})} atom means: $K^\ti{p}_{\ti{a},A,B} \leq_{[]} K^\ti{q}_{\ti{a},A',B'}$,
\end{itemize}
then the monotonicity assumption can be formulated as the following integrity constraint:
\asp{:- less\_active(X, P, Q), not param\_inf(X, P, Q).}
which removes all parametrization results where parameters $K^\aspil{P}_{\aspil{X},A,B}$ and $K^\aspil{Q}_{\aspil{X},A',B'}$ exist such that $A \subset A'$ and $K^\aspil{Q}_{\aspil{X},A',B'} <_{[]} K^\aspil{P}_{\aspil{X},A,B}$, thus violating the monotonicity assumption.
Of course, other assumptions can be formulated in the same way.

This subsection succinctly described how we write ASP programs to represent a model and solve all steps of Thomas' modeling inference.
It finds a particularly interesting application in the enumeration of parameters: all possible parametrizations are generated in separate answer sets, and integrity constraints are formulated to remove those that do not fit the assumptions of admissible parametrizations,
thus reducing the number of interesting parametrizations to be considered in the end.
\end{comment}
