\subsection{The Process Hitting framework}
\label{ssec:PH}

A Process Hitting (PH) (\pref{def:PH}) gathers a finite number of concurrent \emph{processes}
grouped into a finite set of \emph{sorts}.
A process belongs to a unique sort and is noted $a_i$ where $a$ is the
sort and $i$ the identifier of the process within the sort $a$.
At any time, one and only one process of each sort is present; a state of the PH thus corresponds to the set of such processes.
 
The concurrent interactions between processes are defined by a set of
\emph{actions}.
Actions describe the replacement of a process by another of the same sort
conditioned by the presence of at most one other process in the current state.
An action is denoted by $\PHfrappe{a_i}{b_j}{b_k}$, which is read as
``$a_i$ \emph{hits} $b_j$ to make it bounce to $b_k$'',
where $a_i,b_j,b_k$ are processes of sorts $a$ and $b$,
called respectively \emph{hitter}, \emph{target} and
\emph{bounce} of the action.

\begin{definition}[Process Hitting]\label{def:PH}
A \emph{Process Hitting} is a triple $(\PHs,\PHl,\PHa)$:
\begin{itemize}
\item $\PHs \DEF \{a,b,\dots\}$ is the finite set of \emph{sorts};
\item $\PHl \DEF \prod_{a\in\PHs} \PHl_a$ is the set of states with $\PHl_a = \{a_0,\dots,a_{l_a}\}$
the finite set of \emph{processes} of sort $a\in\Sigma$ and $l_a$ a positive integer, with
	$a\neq b\Rightarrow \forall(a_i,b_j)\in\PHl_a\times\PHl_b,a_i\neq b_j$;
\item $\PHa \DEF \{ \PHfrappe{a_i}{b_j}{b_k}, \dots \mid
					(a,b)\in\PHs^2 \wedge (a_i,b_j,b_k)\in \PHl_a\times\PHl_b\times\PHl_b$ \\
	\hspace*{2cm} $\wedge b_j\neq b_k \wedge a=b\Rightarrow a_i=b_j\}$
			is the finite set of \emph{actions}.
\end{itemize}
%$\PHproc$ denotes the set of all processes ($\PHproc \DEF \{ a_i\mid a\in\PHs \wedge a_i\in\PHl_a\}$).
\end{definition}

\noindent
%The sort of a process $a_i$ is referred to as $\PHsort(a_i)=a$ and the set of %sorts present in an action $h\in\PHa$ as  %$\PHsort(h) = \{\PHsort(\PHhitter(h)),\PHsort(\PHtarget(h))\}$.
Given a state $s\in \PHl$, the process of sort $a\in\PHs$ present in $s$ is denoted by $\PHget{s}{a}$, that is the $a$-coordinate of the state $s$.
%If $a_i\in \PHl_a$, we define the notation $a_i\in s \EQDEF \PHget{s}{a}=a_i$.
%For all sort $a \in \PHs$, $\PHdirectpredec{a} \DEF \{b \in \PHs \mid \exists \PHfrappe{b_i}{a_j}{a_k}\in\PHa \}$ is the set of its direct predecessors.
An action $h=\PHfrappe{a_i}{b_j}{b_k} \in \PHa$ is \emph{playable} in $s \in L$ if and only if $\PHget{s}{a}=a_i$ and $\PHget{s}{b} = b_j$.
In such a case, $(s\play h)$ stands for the state resulting from the play of the action $h$ in $s$, with
$\PHget{(s\play h)}{b} = b_k$ and $\forall c \in \PHs, c \neq b, \PHget{(s\play h)}{c} = \PHget{s}{c}$.
%For the sake of clarity, $((s\play h)\play h')$, $h'\in\PHa$ is abbreviated as $(s\play h\play h')$.

\paragraph{Modeling cooperation.}
As described in \cite{PMR10-TCSB}, a \emph{cooperative sort} allows to encode any boolean functions between processes to model cooperation.
It is a standard PH sort that does not involve any special treatment regarding the semantics.
However, it is worth noticing that using a cooperative sort introduces a temporal shift which resulting behavior is an over-approximation of the realization of an instantaneous cooperation.

\begin{example*}
\pref{fig:runningPH} represents a PH $(\PHs,\PHl,\PHa)$ with especially:
$\PHs = \{a,b,c,bc\}$,
$\PHl_a = \{a_0,a_1,a_2\}$,
$\PHl_b = \{b_0, b_1\}$,
$\PHl_c = \{c_0, c_1\}$ and
$\PHl_{bc} = \{bc_{00}, bc_{01}, bc_{10}, bc_{11}\}$.
This example models a BRN where the component $a$ has three qualitative levels, components $b$ and $c$ are boolean and $bc$ is a cooperative sort.
In this BRN, $a$ inhibits $b$ at level $2$ while $b$ and $c$ activate $a$ independently (with direct actions such as
$\PHfrappe{b_0}{a_2}{a_1}$) or through a cooperation (with actions defined with the cooperative sort $bc$, such as
$\PHfrappe{bc_{11}}{a_1}{a_2}$).
Thus, the reachability of $a_2$ and $a_0$ is conditioned by a cooperation involving $b$ and $c$.

\begin{figure}[h]
\centering
\scalebox{1.1}{
\begin{tikzpicture}
\path[use as bounding box] (-2,-5.2) rectangle (7,0.7);

\TSort{(0,0)}{b}{2}{t}
\TSort{(0,-3.8)}{c}{2}{b}
\TSort{(4.5,-3)}{a}{3}{r}

\TSetTick{bc}{0}{00}
\TSetTick{bc}{1}{01}
\TSetTick{bc}{2}{10}
\TSetTick{bc}{3}{11}
% \TSetSortLbcel{bc}{$\neg a\wedge b$}
\TSort{(-0.5,-2)}{bc}{4}{b}

\THit{b_1}{bend right}{bc_0}{.north}{bc_2}
\THit{b_1}{bend right}{bc_1}{.north}{bc_3}
\THit{b_0}{}{bc_2}{.north west}{bc_0}
\THit{b_0}{}{bc_3}{.north west}{bc_1}

\THit{c_0}{}{bc_1}{.south}{bc_0}
\THit{c_0}{}{bc_3}{.south}{bc_2}
\THit{c_1}{}{bc_0}{.south}{bc_1}
\THit{c_1}{}{bc_2}{.south}{bc_3}

\path[bounce, bend right=25]
\TBounce{bc_2}{}{bc_0}{.north east}
\TBounce{bc_3}{}{bc_1}{.north east}
;
\path[bounce, bend left=80, distance=30]
\TBounce{bc_0}{}{bc_2}{.north}
\TBounce{bc_1}{}{bc_3}{.north}
;
\path[bounce, bend right]
\TBounce{bc_0}{}{bc_1}{.west}
\TBounce{bc_2}{}{bc_3}{.west}
;
\path[bounce, bend left]
\TBounce{bc_3}{}{bc_2}{.east}
\TBounce{bc_1}{}{bc_0}{.east}
;

\THit{bc_3}{thick}{a_1}{.north west}{a_2}
\THit{bc_0}{thick,bend right=130, in=305, distance=140}{a_1}{.south east}{a_0}
\path[bounce, bend left=40]
\TBounce{a_1}{thick}{a_2}{.south west}
\TBounce{a_1}{thick}{a_0}{.north east}
;

\THit{b_0}{thick,bend left,out=50,in=150}{a_2}{.west}{a_1}
\THit{b_1}{thick,bend left,out=80,in=70,distance=100}{a_0}{.east}{a_1}
\path[bounce]
\TBounce{a_2}{thick,bend right=40}{a_1}{.west}
\TBounce{a_0}{thick,bend right=40}{a_1}{.east}
;

\THit{c_0}{thick,bend left,out=270,in=290, distance=115}{a_2}{.east}{a_1}
\THit{c_1}{thick,}{a_0}{.north west}{a_1}
\path[bounce]
\TBounce{a_2}{thick,bend left=40}{a_1}{.north east}
\TBounce{a_0}{thick,bend left=40}{a_1}{.south west}
;

\THit{a_2}{bend left, out=290, in=120}{b_1}{.south}{b_0}
\path[bounce, bend left]
\TBounce{b_1}{}{b_0}{.south}
;

\end{tikzpicture}
}

\caption{\label{fig:runningPH}
A PH example with four sorts: three components ($a$, $b$ and $c$) and a cooperative sort ($bc$).
Actions targeting processes of $a$ are in thick lines.
}
\end{figure}
\end{example*}
