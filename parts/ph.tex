\subsection{The Process Hitting framework}
\label{ssec:PH}

A PH (\pref{def:PH}) gathers a finite number of concurrent \emph{processes}
grouped into a finite set of \emph{sorts}.
A process belongs to a unique sort and is noted $a_i$ where $a$ is the
sort and $i$ the identifier of the process within the sort $a$.
At any time, one and only one process of each sort is present; a state of the PH thus corresponds to the set of such processes.
 
The concurrent interactions between processes are defined by a set of
\emph{actions}.
Actions describe the replacement of a process by another of the same sort
conditioned by the presence of at most one other process in the current
state of the PH.
An action is denoted by $\PHfrappe{a_i}{b_j}{b_k}$ where $a_i,b_j,b_k$ are processes
of sorts $a$ and $b$.
An action $h=\PHfrappe{a_i}{b_j}{b_k}$ is read as ``$a_i$ \emph{hits} $b_j$ to
make it bounce to $b_k$'', and
$a_i,b_j,b_k$ are called respectively \emph{hitter}, \emph{target} and
\emph{bounce} of the action, and can be referred to as
$\PHhitter(h), \PHtarget(h), \PHbounce(h)$, respectively.

\begin{definition}[Process Hitting]\label{def:PH}
A \emph{Process Hitting} is a triple $(\PHs,\PHl,\PHa)$:
\begin{itemize}
\item $\PHs \DEF \{a,b,\dots\}$ is the finite set of \emph{sorts};
\item $\PHl \DEF \prod_{a\in\PHs} \PHl_a$ is the set of states with $\PHl_a = \{a_0,\dots,a_{l_a}\}$
the finite set of \emph{processes} of sort $a\in\Sigma$ and $l_a$ a positive integer with
	$a\neq b\Rightarrow \forall(a_i,b_j)\in\PHl_a\times\PHl_b,a_i\neq b_j$;
\item $\PHa \DEF \{ \PHfrappe{a_i}{b_j}{b_k}, \dots \mid
					(a,b)\in\PHs^2 \wedge (a_i,b_j,b_k)\in \PHl_a\times\PHl_b\times\PHl_b$ \\
	\hspace*{2cm} $\wedge b_j\neq b_k \wedge a=b\Rightarrow a_i=b_j\}$
			is the finite set of \emph{actions}.
\end{itemize}
$\PHproc$ denotes the set of all processes ($\PHproc \DEF \{ a_i\mid a\in\PHs \wedge a_i\in\PHl_a\}$).
\end{definition}

\noindent
The sort of a process $a_i$ is referred to as $\PHsort(a_i)=a$ and the set of
sorts present in an action $h\in\PHa$ as 
$\PHsort(h) = \{\PHsort(\PHhitter(h)),\PHsort(\PHtarget(h))\}$.
Given a state $s\in \PHl$, the process of sort $a\in\PHs$ present in $s$ is
denoted by $\PHget{s}{a}$, that is the $a$-coordinate of the state $s$.
If $a_i\in \PHl_a$, we define the notation $a_i\in s \EQDEF \PHget{s}{a}=a_i$.

An action $h=\PHfrappe{a_i}{b_j}{b_k} \in\PHa$ is \emph{playable} in $s\in L$
if and only if $\PHget{s}{a}=a_i$ and $\PHget{s}{b}=b_j$.
In such a case, $(s\play h)$ stands for the state resulting from the play of
the action $h$ in $s$, that is 
$\PHget{(s\play h)}{b} = b_k$ and 
$\forall c\in\PHs, c\neq b, \PHget{(s\play h)}{c} = \PHget{s}{c}$.
For the sake of clarity, $((s\play h)\play h')$, $h'\in\PHa$ is abbreviated as
$(s\play h\play h')$.

\paragraph{Modeling cooperation.}
As described in \cite{PMR10-TCSB}, cooperation between processes to make another process bounce can be
expressed in PH by building a \emph{cooperative sort}.
\pref{fig:runningPH} shows an example of cooperation between processes $b_1$ and $c_1$ to
make $a_1$ bounce to $a_2$:
a cooperative sort $bc$ is defined with 4 processes (one for each sub-state of the presence of
processes $b_1$ and $c_1$), and the process $bc_{11}$ hits $a_1$ instead of the processes
$b_1$ and $c_1$ independently.
We note that cooperative sorts are standard PH sorts and do not involve any
special treatment regarding the semantics of related actions.

While the construction of cooperation in PH allows to encode any boolean functions
between cooperating processes, it is worth noticing they introduce a temporal
shift in their application.
This allows the existence of interleaving of actions leading to a cooperative sort representing a
past sub-state of the presence of the cooperative processes.
The resulting behavior is then an over-approximation of the realization of an instantaneous
cooperation.

\begin{example*}
\pref{fig:runningPH} represents a PH $(\PHs,\PHl,\PHa)$ with especially:
$\PHs = \{a,b,c,bc\}$,
$\PHl_a = \{a_0,a_1,a_2\}$,
$\PHl_b = \{b_0, b_1\}$,
$\PHl_c = \{c_0, c_1\}$ and
$\PHl_{bc} = \{bc_{00}, bc_{01}, bc_{10}, bc_{11}\}$.
%Furthermore, the action $h=\PHfrappe{b_1}{a_1}{a_2}$ is playable in the state
%$s = \state{b_1,a_1,c_0}$; and $s\play h=\state{b_1,a_2,c_0}$.

\begin{comment}
\begin{figure}[t]
\centering
\scalebox{1.3}{
\begin{tikzpicture}
\TSort{(0,0)}{a}{3}{r}
\TSort{(-3,0.5)}{b}{2}{l}
\TSort{(3,0.5)}{c}{2}{r}

\THit{b_1}{very thick}{a_0}{.west}{a_1}
\THit{b_1}{very thick}{a_1}{.north west}{a_2}
\THit{b_0}{}{a_2}{.west}{a_1}
\THit{b_0}{}{a_1}{.west}{a_0}

\path[bounce, bend left=60]
\TBounce{a_1}{very thick}{a_2}{.south}
\TBounce{a_0}{very thick}{a_1}{.south}
;
\path[bounce, bend right=60]
\TBounce{a_2}{}{a_1}{.north}
\TBounce{a_1}{}{a_0}{.north}
;

\THit{c_1}{very thick}{a_0}{.east}{a_1}
\THit{c_1}{very thick}{a_1}{.north east}{a_2}
\THit{c_0}{}{a_2}{.east}{a_1}
\THit{c_0}{}{a_1}{.east}{a_0}

\path[bounce, bend right=60]
\TBounce{a_1}{very thick}{a_2}{.south east}
\TBounce{a_0}{very thick}{a_1}{.south east}
;
\path[bounce, bend left=60]
\TBounce{a_2}{}{a_1}{.north}
\TBounce{a_1}{}{a_0}{.north east}
;

\THit{a_2}{bend right}{b_1}{.north east}{b_0}
\path[bounce, bend left=80]
\TBounce{b_1}{out=100,in=140}{b_0}{.north}
;

\end{tikzpicture}
}
\caption{\label{fig:runningPH-1}
A Process Hitting (PH) example.
Sorts are represented by labeled boxes, and processes by circles (ticks are
the identifiers of the processes within the sort, for instance, $a_0$ is the
process ticked $0$ in the box $a$).
An action (for instance $\PHfrappe{b_1}{a_1}{a_2}$) is represented by a pair of
directed arcs, having the hit part ($b_1$ to $a_1$) in plain line and the bounce
part ($a_1$ to $a_2$) in dotted line.
Actions involving $b_1$ or $c_1$ are in thick lines.
%The current state is represented by the grayed processes:
%$\state{a_0,b_1,c_0,d_0}$.
}
\end{figure}

\end{comment}

This PH example actually models a BRN where the component $a$ has three qualitative
levels and components $b$ and $c$ are boolean.
In this BRN, $a$ inhibits $b$ (when its level is at $2$) while $b$ and $c$ activate $a$ both independently (with direct actions such as
$\PHfrappe{b_0}{a_2}{a_1}$) or through a cooperation (with action defined from the cooperative sort $bc$, such as
$\PHfrappe{bc_{11}}{a_1}{a_2}$).
Hence, this PH expresses a BRN where $a$ requires both $b$ and $c$ active to reach its
highest level, and $a$ does not become inactive unless both $b$ and $c$ are inactive.
\end{example*}
