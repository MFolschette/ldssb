\subsection{The Process Hitting framework}
\label{ssec:PH}

A PH (\pref{def:PH}) gathers a finite number of concurrent \emph{processes}
grouped into a finite set of \emph{sorts}.
A process belongs to a unique sort and is noted $a_i$ where $a$ is the
sort and $i$ the identifier of the process within the sort $a$.
At any time, one and only one process of each sort is present; a state of the PH thus corresponds to the set of such processes.
 
The concurrent interactions between processes are defined by a set of
\emph{actions}.
Actions describe the replacement of a process by another of the same sort
conditioned by the presence of at most one other process in the current state.
An action is denoted by $\PHfrappe{a_i}{b_j}{b_k}$, which is read as
``$a_i$ \emph{hits} $b_j$ to make it bounce to $b_k$'',
where $a_i,b_j,b_k$ are processes of sorts $a$ and $b$,
called respectively \emph{hitter}, \emph{target} and
\emph{bounce} of the action.%, and can be referred to as
%$\PHhitter(h), \PHtarget(h), \PHbounce(h)$, respectively.

\begin{definition}[Process Hitting]\label{def:PH}
A \emph{Process Hitting} is a triple $(\PHs,\PHl,\PHa)$:
\begin{itemize}
\item $\PHs \DEF \{a,b,\dots\}$ is the finite set of \emph{sorts};
\item $\PHl \DEF \prod_{a\in\PHs} \PHl_a$ is the set of states with $\PHl_a = \{a_0,\dots,a_{l_a}\}$
the finite set of \emph{processes} of sort $a\in\Sigma$ and $l_a$ a positive integer, with
	$a\neq b\Rightarrow \forall(a_i,b_j)\in\PHl_a\times\PHl_b,a_i\neq b_j$;
\item $\PHa \DEF \{ \PHfrappe{a_i}{b_j}{b_k}, \dots \mid
					(a,b)\in\PHs^2 \wedge (a_i,b_j,b_k)\in \PHl_a\times\PHl_b\times\PHl_b$ \\
	\hspace*{2cm} $\wedge b_j\neq b_k \wedge a=b\Rightarrow a_i=b_j\}$
			is the finite set of \emph{actions}.
\end{itemize}
%$\PHproc$ denotes the set of all processes ($\PHproc \DEF \{ a_i\mid a\in\PHs \wedge a_i\in\PHl_a\}$).
\end{definition}

\noindent
%The sort of a process $a_i$ is referred to as $\PHsort(a_i)=a$ and the set of %sorts present in an action $h\in\PHa$ as  %$\PHsort(h) = \{\PHsort(\PHhitter(h)),\PHsort(\PHtarget(h))\}$.
Given a state $s\in \PHl$, the process of sort $a\in\PHs$ present in $s$ is denoted by $\PHget{s}{a}$, that is the $a$-coordinate of the state $s$.
%If $a_i\in \PHl_a$, we define the notation $a_i\in s \EQDEF \PHget{s}{a}=a_i$.
%For all sort $a \in \PHs$, $\PHdirectpredec{a} \DEF \{b \in \PHs \mid \exists \PHfrappe{b_i}{a_j}{a_k}\in\PHa \}$ is the set of its direct predecessors.
An action $h=\PHfrappe{a_i}{b_j}{b_k} \in \PHa$ is \emph{playable} in $s \in L$ if and only if $\PHget{s}{a}=a_i$ and $\PHget{s}{b} = b_j$.
In such a case, $(s\play h)$ stands for the state resulting from the play of the action $h$ in $s$, with
$\PHget{(s\play h)}{b} = b_k$ and $\forall c \in \PHs, c \neq b, \PHget{(s\play h)}{c} = \PHget{s}{c}$.
%For the sake of clarity, $((s\play h)\play h')$, $h'\in\PHa$ is abbreviated as $(s\play h\play h')$.

\paragraph{Modeling cooperation.}
As described in \cite{PMR10-TCSB}, a \emph{cooperative sort} allows to encode any boolean functions between processes to model cooperation.
It is a standard PH sort that does not involve any special treatment regarding the semantics.
However, it is worth noticing that using a cooperative sort introduces a temporal shift %by allowing the existence of interleaving of actions leading to a cooperative sort representing a past sub-state of the presence of the cooperative processes.
which resulting behavior is an over-approximation of the realization of an instantaneous cooperation.

\begin{comment}
As described in \cite{PMR10-TCSB}, cooperation between processes to make another process bounce can be
expressed in PH by building a \emph{cooperative sort}. This allows to encode any boolean functions
between cooperating processes using standard PH sorts that do not involve any
special treatment regarding the semantics of related actions.
However, it is worth noticing they introduce a temporal shift in their application
by allowing the existence of interleaving of actions leading to a cooperative sort representing a
past sub-state of the presence of the cooperative processes.
The resulting behavior is then an over-approximation of the realization of an instantaneous
cooperation.
\end{comment}

\begin{example*}
\pref{fig:runningPH} represents a PH $(\PHs,\PHl,\PHa)$ with especially:
$\PHs = \{a,b,c,bc\}$,
$\PHl_a = \{a_0,a_1,a_2\}$,
$\PHl_b = \{b_0, b_1\}$,
$\PHl_c = \{c_0, c_1\}$ and
$\PHl_{bc} = \{bc_{00}, bc_{01}, bc_{10}, bc_{11}\}$.
%Furthermore, the action $h=\PHfrappe{b_1}{a_1}{a_2}$ is playable in the state
%$s = \state{b_1,a_1,c_0}$; and $s\play h=\state{b_1,a_2,c_0}$.
%Sort $bc$ is an example of cooperation between processes $b_1$ and $c_1$ to make $a_1$ bounce to $a_2$:
%a cooperative sort $bc$ is defined with 4 processes (one for each sub-state of the presence of processes $b_1$ and $c_1$), and the process $bc_{11}$ hits $a_1$ instead of the processes $b_1$ and $c_1$ independently.
This example models a BRN where the component $a$ has three qualitative levels, components $b$ and $c$ are boolean and $bc$ is a cooperative sort.
In this BRN, $a$ inhibits $b$ at level $2$ while $b$ and $c$ activate $a$ both independently (with direct actions such as
$\PHfrappe{b_0}{a_2}{a_1}$) or through a cooperation (with action defined the cooperative sort $bc$, such as
$\PHfrappe{bc_{11}}{a_1}{a_2}$).
Hence, because of the cooperative sort $bc$, the reachability of $a_2$ and $a_0$ is conditioned by the active process of both $b$ and $c$.
\end{example*}
