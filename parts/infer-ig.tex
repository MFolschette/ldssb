\section{Interaction Graph Inference}\label{sec:infer-IG}

In order to infer a complete BRN, one has to find the Interaction Graph (IG) first, as some
constraints on the parametrization rely on it.
%Inferring the IG is an abstraction step which consists in determining the global influence of components on each of its successors.
Inferring the IG is an abstraction step which consists, from atomistic actions of a PH, in determining the global influence of components on each of its successors.
We consider hereafter a global PH $(\PHs,\PHl,\PHa)$ on which the IG inference is to be
performed.

We call \emph{context} a set $\ctx$ of processes that are potentially active.
Many of the inferences defined in the rest of this paper rely on the knowledge of \emph{focal
processes} $\focals(a,S_a,\ctx)$ (\pref{def:focals}) amongst a subset $S_a \subset \PHl_a$ of the processes of a sort $a$, w.r.t. a given context $\ctx$ that is possibly strict (\pref{def:strict-ctx}).
If $\focals(a,S_a,\ctx)$ is not empty, we expect, when the context $\ctx$ applies, to always reach one focal process in a bounded number of actions.
If the set of focal processes is empty for a given context, this means that no stable state can be reached in this context in a finite number of actions (due to the presence of a cycle amongst bounces).



\subsection{Well-formed Process Hitting for Interaction Graph Inference}\label{ssec:wf}

The inference of an IG from a PH assumes that the PH defines two types of sorts:
the sorts corresponding to BRN components, and the cooperative sorts.
This leads to the characterization of a \emph{well-formed} PH for IG inference.

The identification of sorts modeling components relies on the observation that their processes
represent (ordered) qualitative levels.
Hence, to respect BRNs dynamics, an action on such a sort cannot make it bounce to a process at a distance more than one.
The set of sorts satisfying such a condition is referred to as $\Gamma$
(\pref{eq:PH-components}), and is therefore the set of components of the BRN to infer.

Any sort that does not act as a component should then be treated as a cooperative sort, whose role is to compute the current state of set of cooperating processes, as explained in \pref{ssec:PH}.
Hence, for each sub-state of its predecessors, a cooperative sort should converge to a unique focal process, as expressed by \pref{pro:wf-cooperative-sort}.
Finally, \pref{pro:wf-ph} sums up the conditions for a Process Hitting to be suitable for IG inference.
In addition of having either component sorts or well-formed cooperative sorts, we also require that there is no cycle between cooperative sorts, and that sorts being never hit (\ie{} serving as an invariant environment) are components.

\begin{example*}
In the PH of \pref{fig:runningPH}, $bc$ is a well-formed cooperative sort as defined in \pref{pro:wf-cooperative-sort}, because:
\begin{align*}
\focals(bc, \PHl_{bc}, \{b_0, c_0\} \cup \PHl_{bc}) = \{bc_{00}\} && \focals(bc, \PHl_{bc}, \{b_0, c_1\} \cup \PHl_{bc}) = \{bc_{01}\} \\
\focals(bc, \PHl_{bc}, \{b_1, c_0\} \cup \PHl_{bc}) = \{bc_{10}\} && \focals(bc, \PHl_{bc}, \{b_1, c_1\} \cup \PHl_{bc}) = \{bc_{11}\}
\end{align*}
Hence, \pref{fig:runningPH} is a well-formed PH for IG inference with $\Gamma = \{a,b,c\}$.
\end{example*}



\subsection{Interaction Inference}\label{ssec:infer-IG}

At this point we can divide the set of sorts $\PHs$ into components ($\Gamma$, see \pref{eq:PH-components}) and cooperative sorts
($\PHs \setminus \Gamma$) that will not appear in the IG. 
We define as $\PHpredec{a}$ (\pref{eq:ph_predec}) the set of \emph{predecessors} of a sort $a$, that is, the sorts influencing $a$
by considering direct actions and possible intermediate cooperative sorts.
The \emph{regulators} of $a$, denoted $\PHpredecgene{a}$ (\pref{eq:regulators}), are its predecessors that are components.

Inferring the underlying IG of a PH consists in finding the influence on a component of all of its regulators.
We aim at inferring that $b$ activates (inhibits) $a$ if there exists a configuration where increasing the level of $b$ makes possible the increase (decrease) of the level of $a$.
This is analogous to standard IG inferences from discrete maps \cite{RiCo07}.

\begin{comment}
Given a set $g$ of components and a configuration (\ie a sub-state) $\sigma$, $\ctx_g(\sigma)$
refers to the set of processes hitting $a$ regulated by any sort in $g$ (\pref{eq:ctx-sigma}).
If $g=\{b\}$, we simple note $\ctx_b(\sigma)$.
This set is composed of the active processes of sorts in $g$, and the focal process (assumed
unique) of the cooperative sorts $\upsilon$ hitting $a$ that have a predecessor in $g$.
The evaluation of the focal process of $\upsilon$ in context $\sigma$, denoted $\upsilon(\sigma)$,
relies on \pref{pro:wf-cooperative-sort}, which gives its value when all the direct predecessors of
$\upsilon$ are defined in $\sigma$.
When a predecessor $\upsilon'$ is not in $\sigma$, we extend the evaluation by recursively computing
the focal value of $\upsilon'$ is $\sigma$, as stated in \pref{eq:cooperative-eval}.
Because there is no cycle between cooperative sorts, this recursive evaluation of $\upsilon(\sigma)$
always terminates.
\end{comment}

We rely on the focal processes given a context that represents a possible configuration of the regulators of a component.
For this, we define the set $\ctx_g(\sigma)$ of processes hitting a sort in a given context, considering also cooperative sorts (\pref{eq:ctx-sigma}).
This reasoning can be straightforwardly applied to a PH when inferring the influence of $b$ on $a$
when $b\neq a$ (\pref{eq:edges-inference-b}), by considering the set $\gamma(b\rightarrow a)$ of components cooperating with $b$ to hit $a$ (\pref{eq:cooperating-with-b}). In the case of a self-influence (when $b = a$) it is required to consider groups of sorts having a cooperation on $a$, denoted $X(a)$ (\pref{eq:influence-groups}). Finally, if we denote $\sigma\{b_i\}$ the configuration $\sigma$ where the process of sort $b$ has been replaced
by $b_i$, then comparing the sets $\focals(a, \{\sigma[a]\}, \ctx_b(\sigma\{b_i\}))$ and $\focals(a, \{\sigma[a]\}, \ctx_b(\sigma\{b_{i+1}\}))$ gives the \emph{atomistic influence} of $b$ on $a$ for the threshold $i$, as detailed in \pref{pps:inference-edges}.

\begin{comment}
Given a configuration $\sigma\in\prod_{c\in\gamma(b\rightarrow a)} L_c$, 
$\focals(a,\{a_i\},\ctx_b(\sigma))$ gives the bounces that a given process $a_i$ can make in the
context $\ctx_b(\sigma)$.
We note $\sigma\{b_i\}$ the configuration $\sigma$ where the process of sort $b$ has been replaced
by $b_i$.
If there exists $b_i,b_{i+1}\in L_b$ such that one bounce in 
$\focals(a, \{\sigma[a]\}, \ctx_b(\sigma\{b_i\}))$
has a lower (resp. higher) level that one bounce in
$\focals(a, \{\sigma[a]\}, \ctx_b(\sigma\{b_{i+1}\}))$, then
$b$ as positive (resp. negative) influence on $a$ with a maximum threshold $l=i+1$.

Then, we infer that $a$ has a self-influence if its current level can have an impact on its own
evolution at a given configuration $\sigma$.
We consider here a configuration $\sigma$ of a group $g$ of sorts having a cooperation on $a$.
This set of sort groups is given by $X(a)$ (\pref{eq:influence-groups}) which returns the set of
connected components (noted $\mathcal C$) of the graph linking two regulators $b,c$ of $a$ if there
is a cooperative sort hitting $a$ regulated by both of them.
Given $a_i,a_{i+1}\in L_a$, we pick $a_j\in\focals(a,\{a_i\},\ctx_g(\sigma\{a_i\}))$ and
$a_k\in\focals(a,\{a_{i+1}\},\ctx_g(\sigma\{a_{i+1}\}))$.
If $k=j+1$, we can not conclude as there is no difference in the evolution of both levels.
If $k\neq j+1$ and $k-j\neq 0$, then $a_i$ and $a_{i+1}$ have divergent evolutions: we infer an
influence of sign of $k-j$ at threshold $i+1$.
We note that some aspects of this inference are arbitrary and may impact the number of parameters to
infer in the next section.
In particular, in some cases, the use of intervals for Thomas' parameters drops the requirement of
inferring a self-activation.
%Future work may propose alternative definitions of self-influences inference in order to range over
%different parametrization configurations.
\end{comment}

We are then able to infer the edges of the final IG by considering global influences (\pref{pps:inference-IG}).
We infer a positive (resp. negative) edge if there exists a corresponding atomistic influence with the same
sign. If a global influence is both positive and negative, we infer an unsigned edge. In the end, the
threshold of each edge is the minimum threshold for which an atomistic influence has been found. As unsigned
edges represent ambiguous interactions, no threshold is inferred.

\begin{example*}
The IG inference from the PH of \pref{fig:runningPH} gives
$\hat{E}_+ = \{b \xrightarrow{1} a, c \xrightarrow{1} a\}$ and 
$\hat{E}_- = \{a \xrightarrow{2} b\}$, corresponding to the IG of \pref{fig:runningBRN}.
No self-influence are inferred ($X(a) = \{ \{b,c\} \}$, $X(b)=\{ \{a\}\}$, and $X(c)=\emptyset$).
\end{example*}
