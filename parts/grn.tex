\subsection{Thomas' modeling}
Thomas' formalism, here inspired by \cite{Richard06,BernotSemBRN}, lies on two complementary descriptions of the system.
First, the \emph{Interaction Graph} (IG) models the structure of the system by defining the components' mutual influences.
Its nodes represent components, while its edges labeled with a threshold stand for either positive or negative interactions (\pref{def:ig});
$l_a$ denotes the maximum level of any component $a$.
%We allow any number of levels for the components, regardless of the number of outgoing edges, in order to be as general as possible.

\begin{definition}[Interaction Graph]
\label{def:ig}
An \emph{Interaction Graph} (IG) is a triple $(\Gamma, E_+, E_-)$ where $\Gamma$ is a finite number of \emph{components},
and $E_+$ (resp. $E_-$) $\subset \{a \xrightarrow{t} b \mid a, b \in \Gamma \wedge t \in [1; l_a]\}$
is the set of positive (resp. negative) \emph{regulations} between two nodes, labeled with a \emph{threshold}.

A regulation from $a$ to $b$ is uniquely referenced:
if $a \xrightarrow{t} b \in E_+$ (resp. $E_-$),
$\nexists a \xrightarrow{t'} b \in E_+ \text{ (resp. $E_-$)}, t' \neq t$
and $\nexists a \xrightarrow{t'} b \in E_-\text{ (resp. $E_+$)}, t' \in \mathbb{N}$.
\end{definition}

\begin{comment}
\begin{definition}[Effective levels ($\levels$)]\label{def:levels}
Let $(\Gamma,E_+,E_-)$ be an IG and $a, b \in \Gamma$ two of its components:
\begin{itemize}
  \item if $a \xrightarrow{t} b \in E_+$, $\levelsA{a}{b} \DEF [t; l_a]$ and
    $\levelsI{a}{b} \DEF [0; t-1]$;
  \item if $a \xrightarrow{t} b \in E_-$, $\levelsA{a}{b} \DEF [0; t-1]$ and
    $\levelsI{a}{b} \DEF [t; l_a]$;
  \item otherwise, $\levelsA{a}{b} \DEF \levelsI{a}{b} \DEF \emptyset$.
\end{itemize}
\end{definition}
\end{comment}

\noindent
For an interaction of the IG to take place, the expression level of its head component has to be higher than its threshold; otherwise, the opposite influence is expressed.
%Thus, we call $\levelsA{a}{b}$ (resp. $\levelsI{a}{b}$) the levels of $a$ where it is an activator (resp. inhibitor) of $b$ (\pref{def:levels}).
For all component $a \in \Gamma$, $\GRNreg{a} \DEF \{ b\in\Gamma\mid \exists b\xrightarrow t a\in E_+\cup E_- \}$
is the set of its regulators.
A \emph{state} $s$ of an IG $(\Gamma, E_+, E_-)$ is an element in $\prod_{a \in \Gamma} [0;l_a]$
and $\GRNget{s}{a}$ refers to the level of component $a$ in $s$.

%Then, the \emph{parametrization} specifies the levels to which tends a component when a given configuration of its regulators applies.
The specificity of Thomas' approach then lies in the use of discrete \emph{parameters} to represent the focal level intervals (\pref{def:param}).
% towards which a component will evolve in each configuration of its regulators.
%Indeed, for each possible state of a BRN, all regulators of a component $a$ can be divided into \emph{activators} and \emph{inhibitors}, given their type of interaction and expression level.
%, referred to as the \emph{resources} of $a$ in this state (\pref{def:resources}).
While the use of intervals as parameters does not add expressivity in boolean networks, it allows to specify a larger range of dynamics in the general case (w.r.t. to a fixed IG).
Indeed, if a parameter is an interval of three values, it is impossible to split the three cases with the boolean definition of resources.

\begin{definition}[Discrete parameter $K_{a,A,B}$ and Parametrization $K$]\label{def:param}
For a given component $a \in \Gamma$ and $A$ (resp. $B$) $\subset \GRNreg{a}$ a set of its \emph{activators} (resp. \emph{inhibitors}) such that $A \cup B = \GRNreg{a}$ and $A \cap B = \emptyset$,
the discrete \emph{parameter} $K_{a,A,B} = [i; j]$ is a non-empty interval towards which $a$ will tend
in the states where its activators (resp. inhibitors) are the regulators in set $A$ (resp. $B$).
The complete map $K$ of discrete parameters for $\IG$ is called a \emph{parametrization} of $\IG$.
\end{definition}

\begin{comment}
\begin{definition}[Resources $\GRNres{a}{s}$]\label{def:resources}
For a given state $s$ of a BRN, we define the \emph{activators} $A$ and \emph{inhibitors} $B$ of $a$ in $s$ as $\GRNres{a}{s} = A,B$, where:
\begin{align*}
  A &= \{b \in \Gamma \mid \GRNget{s}{b} \in \levelsA{b}{a}\} \\
  B &= \{b \in \Gamma \mid \GRNget{s}{b} \in \levelsI{b}{a}\}
\end{align*}
We also denote: $\GRNallres{a} = \{(A;B) \mid \exists s \in \textstyle\prod_{a \in \Gamma} [0;l_a], \GRNres{a}{s} = A,B\}$
\end{definition}

%\begin{example*}
%\pref{fig:runningBRN}(right) gives a Parametrization of the IG of \pref{fig:runningBRN}(left).
%\end{example*}
\end{comment}

%At last, \pref{def:dynamics} gives the asynchronous dynamics of a BRN using Thomas' parameters.
At last, dynamics are defined in BRN in an unitary and asynchronous way:
from a given state $s$, a transition to another state $s'$ is possible provided that only one component $a$ will evolve of exactly one level towards $K_{a,A,B}$, where $A$ (resp. $B$) is the set of activators (resp. inhibitors) of $a$ in $s$.

\begin{example*}
\pref{fig:runningBRN}(left) represents an Interaction Graph $(\Gamma,E_+,E_-)$ with
$\Gamma = \{a, b, c\}$,
$E_+ = \{b \xrightarrow{1} a, c \xrightarrow{1} a\}$ and
$E_- = \{a \xrightarrow{2} b\}$;
hence $\GRNreg{a} = \{b, c\}$.
\pref{fig:runningBRN}(right) gives a possible parametrization on this IG.
In this BRN, the following transitions are possible:
$\GRNetat{a_0, b_1, c_1} \rightarrow \GRNetat{a_1, b_1, c_1} \rightarrow \GRNetat{a_2, b_1, c_1} \rightarrow
\GRNetat{a_2, b_0, c_1} \rightarrow \GRNetat{a_1, b_0, c_1}$,
where $a_i$ is the component $a$ at level $i$.
\end{example*}

\begin{figure}[h]
\begin{minipage}{0.4\linewidth}
\centering
\scalebox{1.2}{
\begin{tikzpicture}[grn]
\path[use as bounding box] (-0.3,-0.75) rectangle (2.5,1.2);
\node[inner sep=0] (a) at (2,0) {a};
\node[inner sep=0] (b) at (0,0) {b};
\node[inner sep=0] (c) at (2,1.2) {c};
%\path
%  node[elabel, below=-1em of a] {$0..2$}
%  node[elabel, below=-1em of b] {$0..1$}
%  node[elabel, below=-1em of c] {$0..1$};
\path[->]
  (b) edge[bend right] node[elabel, below=-2pt] {$+1$} (a)
  (c) edge node[elabel, right=-2pt] {$+1$} (a)
  (a) edge[bend right] node[elabel, above=-5pt] {$-2$} (b);
\end{tikzpicture}
}
\end{minipage}
\begin{minipage}{0.6\linewidth}
\centering
\begin{align*}
K_{a,\{b,c\},\emptyset} &= [2 ; 2] & K_{b,\{a\},\emptyset} &= [0 ; 1] \\
K_{a,\{b\},\{c\}} &= [1 ; 1] & K_{b,\emptyset,\{a\}} &= [0 ; 0] \\
K_{a,\{c\},\{b\}} &= [1 ; 1] &&\\
K_{a,\emptyset,\{b,c\}} &= [0 ; 0] & K_{c,\emptyset,\emptyset} &= [0 ; 1]
\end{align*}
\end{minipage}
\caption{\label{fig:runningBRN}
(left)
IG example.
Regulations are represented by the edges labeled with their sign and threshold.
For instance, the edge from $b$ to $a$ is labeled $+1$, which stands for: $b \xrightarrow{1} a \in
E_+$.
(right)
Example parametrization of the left IG.
}
\end{figure}
