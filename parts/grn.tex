\subsection{Thomas' modeling}
Thomas' formalism, here inspired by \cite{Richard06,BernotSemBRN}, lies on two complementary descriptions of the system.
First, the \emph{Interaction Graph} (IG) models the structure of the system by defining the components' mutual influences.
Its nodes represent components, while its edges labeled with a threshold stand for either positive or negative interactions (\pref{def:ig});
$l_a$ denotes the maximum level of a component $a$.

\begin{definition}[Interaction Graph]
\label{def:ig}
An \emph{Interaction Graph} (IG) is a triple $(\Gamma, E_+, E_-)$ where $\Gamma$ is a finite number of \emph{components},
and $E_+$ (resp. $E_-$) $\subset \{a \xrightarrow{t} b \mid a, b \in \Gamma \wedge t \in [1; l_a]\}$
is the set of positive (resp. negative) \emph{regulations} between two nodes, labeled with a \emph{threshold}.

A regulation from $a$ to $b$ is uniquely referenced:
if $a \xrightarrow{t} b \in E_+$ (resp. $E_-$),
then $\nexists a \xrightarrow{t'} b \in E_+ \text{ (resp. $E_-$)}, t' \neq t$
and $\nexists a \xrightarrow{t'} b \in E_-\text{ (resp. $E_+$)}, t' \in \mathbb{N}$.
\end{definition}

\noindent
For an interaction of the IG to take place, the expression level of its head component has to be higher than its threshold; otherwise, the opposite influence is expressed.
For any component $a \in \Gamma$, $\GRNreg{a} = \{ b\in\Gamma\mid \exists b\xrightarrow t a\in E_+\cup E_- \}$
is the set of its regulators.
A \emph{state} $s$ of an IG $(\Gamma, E_+, E_-)$ is an element in $\prod_{a \in \Gamma} [0;l_a]$
and $\GRNget{s}{a}$ refers to the level of component $a$ in $s$.

Then, the specificity of Thomas' approach lies in the use of discrete \emph{parameters} to represent focal level intervals (\pref{def:param}).
While the use of intervals as parameters does not add expressivity in Boolean networks, it allows to specify a larger range of dynamics in the general case (w.r.t. a fixed IG).
%Indeed, if a parameter is an interval of three values, it is impossible to split the three cases with the boolean definition of resources.

\begin{definition}[Discrete parameter $K_{x,A,B}$ and Parametrization $K$]\label{def:param}
Let $x \in \Gamma$ be a given component and $A$ (resp. $B$) $\subset \GRNreg{x}$ a set of its \emph{activators} (resp. \emph{inhibitors}),
such that $A \cup B = \GRNreg{a}$ and $A \cap B = \emptyset$.
The discrete \emph{parameter} $K_{x,A,B} = [i; j]$ is a non-empty interval so that $0 \leq i \leq j \leq l_x$.
With regard to the dynamics, $x$ will tend towards $K_{x,A,B}$ in the states where its activators (resp. inhibitors) are the regulators in set $A$ (resp. $B$).
The complete map $K = (K_{x,A,B})_{x,A,B}$ of discrete parameters for an IG is called a \emph{parametrization} of this IG.
\end{definition}

At last, dynamics are defined in BRN in an unitary and asynchronous way:
from a given state $s$, a transition to another state $s'$ is possible provided that only one component $a$ will evolve of exactly one level towards $K_{a,A,B}$, where $A$ (resp. $B$) is the set of activators (resp. inhibitors) of $a$ in $s$.

\begin{example*}
\pref{fig:runningBRN}(left) represents an Interaction Graph $(\Gamma,E_+,E_-)$ with
$\Gamma = \{a, b, c\}$,
$E_+ = \{b \xrightarrow{1} a, c \xrightarrow{1} a\}$ and
$E_- = \{a \xrightarrow{2} b\}$;
hence $\GRNreg{a} = \{b, c\}$.
\pref{fig:runningBRN}(right) gives a possible parametrization on this IG.
In this BRN, the following transitions are possible:
$\GRNetat{a_0, b_1, c_1} \rightarrow \GRNetat{a_1, b_1, c_1} \rightarrow \GRNetat{a_2, b_1, c_1} \rightarrow
\GRNetat{a_2, b_0, c_1} \rightarrow \GRNetat{a_1, b_0, c_1}$,
where $a_i$ is the component $a$ at level $i$.
\end{example*}

\begin{figure}[t]
\begin{minipage}{0.4\linewidth}
\centering
\scalebox{1.2}{
\begin{tikzpicture}[grn]
\path[use as bounding box] (0,-0.7) rectangle (3.5,0.7);
\node[inner sep=0] (a) at (2,0) {a};
\node[inner sep=0] (b) at (0,0) {b};
\node[inner sep=0] (c) at (3.5,0) {c};
%\path
%  node[elabel, below=-1em of a] {$0..2$}
%  node[elabel, below=-1em of b] {$0..1$}
%  node[elabel, below=-1em of c] {$0..1$};
\path[->]
  (b) edge[bend right] node[elabel, below=-3pt] {$+1$} (a)
  (c) edge node[elabel, above=-5pt] {$+1$} (a)
  (a) edge[bend right] node[elabel, above=-5pt] {$-2$} (b);
\end{tikzpicture}
}
\end{minipage}
\begin{minipage}{0.6\linewidth}
\centering
\begin{align*}
K_{a,\{b,c\},\emptyset} &= [2 ; 2] & K_{b,\{a\},\emptyset} &= [0 ; 1] \\
K_{a,\{b\},\{c\}} &= [1 ; 1] & K_{b,\emptyset,\{a\}} &= [0 ; 0] \\
K_{a,\{c\},\{b\}} &= [1 ; 1] &&\\
K_{a,\emptyset,\{b,c\}} &= [0 ; 0] & K_{c,\emptyset,\emptyset} &= [0 ; 1]
\end{align*}
\end{minipage}
\caption{\label{fig:runningBRN}
(left)
IG example.
Regulations are represented by the edges labeled with their sign and threshold.
For instance, the edge from $b$ to $a$ is labeled $+1$, which stands for: $b \xrightarrow{1} a \in
E_+$.
(right)
Example parametrization of the left IG.
}
\end{figure}
